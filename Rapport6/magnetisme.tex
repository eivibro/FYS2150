
\documentclass[a4paper,11pt, twocolumn]{article}
\usepackage[utf8]{inputenc}
\usepackage[T1]{fontenc}
\usepackage[norsk]{babel}
\usepackage{graphicx} %for å inkludere grafikk
\usepackage{verbatim} %for å inkludere filer med tegn LaTeX ikke liker
\usepackage{mathpazo}
\usepackage{mathtools}
\usepackage{csquotes}
\usepackage{tikz}
\usepackage{listings}
\usepackage{booktabs}
\usepackage{todonotes}
\usepackage[backend=biber]{biblatex}
\usepackage{caption} 
\usepackage{parboxx}
\usetikzlibrary{decorations.markings}
\hyphenpenalty=750
\captionsetup[table]{skip=10pt}
\addbibresource{magnetisering.bib}

\lstset{language=Matlab, commentstyle=\textcolor[rgb]{0.00,0.50,0.00}, keepspaces=true, columns=flexible, basicstyle=\footnotesize, keywordstyle=\color{blue}, showstringspaces=false, inputencoding=ansinew}

\title{Magnetisme\\ FYS2150}

\author{Eivind Brox}
\date{\today}

\begin{document}
\tikzset{->-/.style={decoration={
  markings, 
  mark=at position #1 with {\arrow[scale=3]{stealth reversed}}},  postaction={decorate}}}
\maketitle

\tikzset{-<-/.style={decoration={
  markings, 
  mark=at position #1 with {\arrow[scale=3]{>}}},  postaction={decorate}}}
\maketitle


\begin{abstract}
\end{abstract}

\section{Introduksjon}
Det er fire partielle ligninger som ligger til grunn for den klassiske elektromagnetiske teorien. Matematisk er Maxwells ligninger, som de blir kalt, fullt symmetriske i forhold til magnetiske og elektriske fenomener. I virkeligheten finner vi likevel små forskjeller som kommer av at det finnes elektriske, men ikke magnetiske monopoler.

Magnetfelt dannes av elektriske ladninger i bevegelse. Disse mikroskopiske bevegelsene danner grunnlaget for lukkede magnetfeltlinjer som vi kan observere effekten av i det makroskopiske perspektiv.
\section{Teori}
En tidlig form av Maxwells ligninger ble publisert mellom 1861 og 1862, og idag skrives de gjerne på differential form~\cite[kap. 9.3.3]{griffithsED}
i
\begin{align}
	&\nabla\cdot\mathbf{E}= \frac{\rho}{\epsilon_0}
	\label{eq:max1}\\
	&\nabla\cdot\mathbf{B} = 0
	\label{eq:max2}\\
	&\nabla\times\mathbf{E} = -\frac{\partial \mathbf{B}}{\partial t}
	\label{eq:max3}\\
	&\nabla\times\mathbf{B} = \mu_0\mathbf{J}+\mu_0\epsilon_0\frac{\partial\mathbf{E}}{\partial t}
	\label{eq:max4}
\end{align}

Her er $\mathbf{E}$ og $\mathbf{B}$ henholdsvis det elektriske og det magnetiske vektorfeltet, $\mu_0$ er den magnetiske permieabiliteten og $\epsilon_0$ er den elektiske permititiviteten $\rho$ er ladningstetthet i vakuum. Strømtetthetten $\mathbf{J}$ er definert slik at 
\begin{equation}
	I = \int_\Omega \mathbf{J}\cdot d\mathbf{a}
	\label{eq:currentDensity}
\end{equation}

der $I$ er den totale strømmen gjennom en lukket flate $\Omega$ med den infinitesimale flateelementvektoren $d\mathbf{a}$ normalt på flatens tangentplan, pekende i motsatt retning av krumningsradiens sentrum. 
\subsection{Magnetfelt gjennom spole}
Betrakter vi ligning~\eqref{eq:max4} og antar at det elektriske feltet ikke endrer seg med tid, så sitter vi igjen med 
\begin{equation}
	\nabla\times\mathbf{B} = \mu_0\mathbf{J}
	\label{eq:independentOfTime}
\end{equation}
Vi kan nå benytte Stokes teorem til å gjøre om integralet slik at
\begin{align}
	\int_\Omega (\nabla\times\mathbf{B})\cdot d\mathbf{a} = \oint_\mathcal{C}\mathbf{B}\cdot d\mathbf{r}=\mu_0\int_\Omega \mathbf{J} \cdot d\mathbf{a} 
 	\label{}
\end{align}
der $d\mathbf{r}$ er den infenitesimale tangentvektoren til kurven $\mathcal{C}$. 

Benytter vi ligning~\eqref{eq:currentDensity} så sitter vi igjen med 

\begin{align}
	\oint_\mathcal{C}\mathbf{B}\cdot d\mathbf{r}=\mu_0 I_\text{omfavnet}
	\label{eq:etterStokes}
\end{align}
der $I_\text{omfavnet}$ er den totalle strømmen som går gjennom flaten omsluttet av den lukkede kurven $\mathcal{C}$.

Vi ser nå på en spole, et tilfelle som er skissert i figur~\ref{fig:spole}. Det magnetiske feltet er tilnærmet null utenfor spolen, og strømmen som går gjennom flaten omsluttet av den lukkede kurven er antall vindinger, $N$, multiplisert med strømmen vi har gjennom spolen.

Siden vi bare har bidrag til det magnetiske feltet langs symmetriaksen til spolen, så reduseres integralet til enkel multiplikasjon. Vi sitter igjen med
\begin{equation}
	BL = \mu_0IN \implies B = \frac{\mu_0 I N}{L}
	\label{eq:magSpole}
\end{equation}

\begin{figure}[!ht]
	\centering
\begin{tikzpicture}
    % Define a formula for the coil.
    % This is what the numbers mean:
    % 0.3 ... how far the rings are apart
    % 0.4 ... how much from the side the rings are seen (try 0 and the same as the radius)
    % 1.5 ... radius of the rings
    \def\coil#1{
        {0.3 * (2*#1 + \t) + 0.4*sin(\t * pi r))},
        {1.5 * cos(\t * pi r)}
        }

    % Draw the part of the coil behind the rectangle
    \foreach \n in {0,1,...,10} {
       \draw[domain={0:1},smooth,variable=\t,samples=15]
            plot (\coil{\n}); 
        }

    % Draw the rectangle
	\draw[-<-=0.80, dashed](-0.5,-2) rectangle (7,0);
	\node (curve) at (3.5, -2.5) {$\mathcal{C}$};

    % Draw the part of the coil in front of the rectangle
\draw[->-=0.3, domain={1:2},smooth,variable=\t,samples=15,
              preaction={draw,white,line width=3pt}     % remove if undesired
      ] node at (-0.3, -0.5)  {$I$}
            plot (\coil{0});
    \foreach \n in {1,2,...,10} {
        \draw[domain={1:2},smooth,variable=\t,samples=15,
              preaction={draw,white,line width=3pt}     % remove if undesired
             ]
            plot (\coil{\n});
        }
	\draw[very thick, ->] (2.5,0) -- (4.5,0);
	\node (field) at (3.9, 0.3) {$\mathbf{B}$};
	\node (surface) at (6, -1.7) {$\Omega$};
\end{tikzpicture}

\caption{Figur som viser spole og valgt integrasjonskurve for ligning~\ref{eq:etterStokes}. Integrasjonskurven ligger langs symmetriaksen til spolen.} 
	\label{fig:spole}
\end{figure}
\section{Eksperimentelt}
\subsection{Diamagnetisme}
\subsection{Ferromagnetrisme}
\section{Resultater}
\section{Diskusjon}

\printbibliography
\clearpage
\onecolumn
\appendix

\section{Vedlegg}
\end{document}




